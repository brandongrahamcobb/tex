\documentclass[10pt, roman]{article}
\usepackage{amsmath}
\usepackage{graphicx}
\usepackage{hyperref}
\usepackage{geometry}
\usepackage{caption}
\geometry{margin=1in}

\graphicspath{{/home/spawd/Pictures/}}

\title{CHEM 1060 Placement Test Guide}
\author{
Brandon Cobb, Clare Williams \\
Academic Success Center, Macomb Community College \\
\textbf{Center Campus}: 44575 Garfield Rd, Clinton Twp, MI 48038 \\
\textbf{South Campus}: 14500 12 Mile Rd, Warren, MI 48088
}

\begin{document}

\maketitle

\tableofcontents

\section{General Information}
\subsection{Placement Tests}
{\centering https://my.macomb.edu/search/?q=placement+tests\par}
Placement tests are available for students at Macomb Community College who have expertise in Chemistry, Physics and Foreign Lanugages.
Macomb Community College has a placement test for CHEM-1060.
To demonstrate proficiency in CHEM-1050 and place into a higher level class, a student can opt to take the placement test.
\newpage
\subsection{Testing Procedure}
\begin{figure}[ht!]
\centering
\includegraphics[width=\textwidth]{ThePeriodicTable.pdf}
\caption{Testing Materials}
\label{overflow}
\end{figure}
Students may take this exam anytime during the posted Academic Success Center hours of operation.
No appointment is required must arrive no later than 60 minutes before closing time.
There is no fee and students are allowed 1 attempt per lifetime to complete the CHEM-1060 placement test.
Students must have photo identification and supply their Macomb student identification number.
Blank scrap paper, a scantron, a periodic table of elements and a scientific calculator will be provided with the exam.
The exam is 44 questions and must be completed in 45 minutes.
\subsection{Passing Score}
A minimum of 25 correct answers are required to achieve a passing score.
Students who pass the exam will be able to register for CHEM-1060 Introduction to Organic Chemisty \& Biochemistry within 2 business days.

\section{Topics}
\subsection{Acid/base Chemistry}
\subsection{Areas of the Periodic Table}
\subsection{Balancing Chemical Equations}
It can be tricky at first, but once you get the hang of it, you should be good to go!
\subsection{Calculating with Avogadro's Number}
\subsection{Calculating with Percentages}
Calculating with percentages may seem confusing at first, but there is a simple trick to them!
\subsection{Collision Theory}
\subsection{Concentration Calculations}
\subsection{Density}
\subsection{Dilution}
\subsection{Electrolytes}
\subsection{Electron Configurations}
\subsection{Energy Level Transitions in Atoms}
\subsection{Entropy of a System}
\subsection{Finding Molecular Weight}
\subsection{Identifying Diatomic Elements}
\subsection{Identifying Isotopes}
\subsection{Identifying Types of Reactions}
\subsection{Ionization Energy}
\subsection{Lewis Structures}
\subsection{Molarity}
\subsection{Naming Ionic Compounds}
\subsection{Nuclide Symbols}
\subsection{Percent Composition}
\subsection{pH}
\subsection{Phase Diagrams}
\subsection{Precision vs. Accuracy}
\subsection{Significant Figures}
\subsection{States of Matter}
\subsection{Stoichiometry Calculations}
\subsection{Types of Bonding}
\subsection{Understanding Graphical Data}
\newpage
\subsection{Unit Conversions}
\begin{minipage}{.5\textwidth}
\centering
\begin{tabular}{|c|}
\hline
\textbf{Length Conversion Factors} \\
\hline
1 Angstrom (Å) = 0.1 nanometers (nm) \\
1 inch (in) = 2.54 centimeters (cm) \\
1 foot (ft) = 0.3048 meters (m) \\
1 mile (mi) = 1.60934 kilometers (km) \\
\hline
\end{tabular}
\end{minipage}%
\begin{minipage}{.5\textwidth}
\centering
\begin{tabular}{|c|}
\hline
\textbf{Energy Conversion Factors} \\
\hline
1 calorie (cal) = 4.184 joules (J) \\
1 kilocalorie (kcal) = 4184 joules (J) \\
\hline
\end{tabular}
\end{minipage}

\vspace{10pt}

\noindent
\begin{minipage}{.5\textwidth}
\centering
\begin{tabular}{|c|c|c|}
\hline
\multicolumn{3}{|c|}{\textbf{Common Metric Prefixes}} \\
\hline
Prefix & Symbol & Factor \\
\hline
Tera & T & $10^{12}$ \\
Giga & G & $10^{9}$ \\
Mega & M & $10^{6}$ \\
Kilo & k & $10^{3}$ \\
Hecto & h & $10^{2}$ \\
Deka & da & $10^{1}$ \\
\hline
(Base Unit) &  & $10^{0}$ \\
\hline
Deci & d & $10^{-1}$ \\
Centi & c & $10^{-2}$ \\
Milli & m & $10^{-3}$ \\
Micro & µ & $10^{-6}$ \\
Nano & n & $10^{-9}$ \\
Pico & p & $10^{-12}$ \\
Femto & f & $10^{-15}$ \\
Atto & a & $10^{-18}$ \\
\hline
\end{tabular}
\end{minipage}
\begin{minipage}{.5\textwidth}
\centering
\begin{tabular}{|c|}
\hline
\textbf{Volume Conversion Factors} \\
\hline
1 teaspoon (tsp) = 4.92892 milliliters (mL) \\
1 tablespoon (tbsp) = 14.7868 milliliters (mL) \\
1 fluid ounce (fl oz) = 29.5735 milliliters (mL) \\
1 cup = 0.236588 liters (L) \\
1 pint (pt) = 0.473176 liters (L) \\
1 quart (qt) = 0.946353 liters (L) \\
1 gallon (gal) = 3.78541 liters (L) \\
\hline
\end{tabular}

\vspace{10pt}

\noindent
\begin{tabular}{|c|}
\hline
\textbf{Mass Conversion Factors} \\
\hline
1 atomic mass unit (amu) = 1 dalton (Da) \\
1 Dalton (Da) $\approx 1.66 \times 10^{-21}$ milligrams (mg) \\
1 ounce (oz) = 28.3495 grams (g) \\
1 pound (lb) = 0.453592 kilograms (kg) \\
1 tonne (ton) = 2000 pounds (lb) \\
\hline
\end{tabular}

\vspace{10pt}

\noindent
\begin{tabular}{|c|}
\hline
\textbf{Pressure Conversion Factors} \\
\hline
1 atmosphere (atm) = 101.325 kilopascals (kPa) \\
1 atmosphere (atm) = 760 mm of Hg (mmHg) \\
\hline
\end{tabular}
\end{minipage}

\vspace{10pt}

\noindent
\begin{minipage}{.5\textwidth}
\centering
\begin{tabular}{|c|c|}
\hline
\multicolumn{2}{|c|}{\textbf{Constants}} \\
\hline
Avogadro’s \# & $N_A = 6.022 \times 10^{23} \text{ mol}^{-1}$ \\
Gas Constant & $R = 8.314 \frac{J}{\text{mol K}}$ \\
Speed of Light & $c = 2.998 \times 10^{8} \frac{m}{s}$ \\
Elementary Charge & $e = 1.602 \times 10^{-19} \text{ C}$ \\
Electron Mass & $m_e = 9.109 \times 10^{-31} \text{ kg}$ \\
Proton Mass & $m_p = 1.673 \times 10^{-27} \text{ kg}$ \\
\hline
\end{tabular}
\end{minipage}%
\begin{minipage}{.5\textwidth}
\centering
\begin{tabular}{|c|}
\hline
\textbf{Time Conversions} \\
\hline
1 minute (min) = 60 seconds (s) \\
1 hour (hr) = 60 minutes (min) \\
1 day (day) = 24 hours (hr) \\
1 week (wk) = 7 days (day) \\
1 month (mo) $\approx 30.44$ days (day) \\
1 year (yr) $\approx 365.25$ days (day) \\
\hline
\end{tabular}
\end{minipage}

\newpage
\subsection{notes}



\subsection{Conversion Equations}

no Graham
both campus addresses
what is chem 1060 paragraph first sentence
prerequisite course -> chem 1050
remove calif test
Image not figure
change to academic success center
also provided with a scantron
link to macomb website
combine finding molecular weight and elements
get rid of dimensional analysis
dim anal take up whole page remove captions

readd reactants and products definition in the types of chemical reactions section


linear ph scal for 1060 
follow naming scheme
Acid–base indicator A compound that exhibits different colors depending on the pH of its solution.
Acid–base neutralization chemical reaction A chemical reaction between an acid and a hydroxide base in which a salt and water are the products.
Acid–base titration A neutralization reaction in which a measured volume of an acid or a base of known concentration is completely reacted with a measured volume of a base or an acid of unknown concentration.
Acid ionization constant (Ka) The equilibrium constant for the reaction of a weak acid with water.
Acidic hydrogen atom A hydrogen atom in an acid molecule that can be transferred to a base in an acid-base reaction.
Acidic solution An aqueous solution in which the concentration of H3O ion is higher than that of OH ion; an aqueous solution whose pH is less than 7.0.
Actual yield The amount of product actually obtained from a chemical reaction.
Alkali metal A general name for any element in Group IA of the periodic table excluding hydrogen.
Alkaline earth metal A general name for any element in Group IIA of the periodic table.
Aqueous solution A solution in which water is the solvent.
Arrhenius, Svante August
Arrhenius acid A hydrogen-containing compound that, in water, produces hydrogen ions (H ions).
Arrhenius base A hydroxide-containing compound that, in water, produces hydroxide ions (OH ions).
Atom The smallest particle of an element that can exist and still have the properties of the element.
Atomic mass The calculated average mass for the isotopes of an element expressed on a scale where 12C serves as the reference point.
Atomic number The number of protons in the nucleus of an atom.
Avogadro, Amedeo
Avogadro’s number The name given to the numerical value 6.02 \times 10^23.
Balanced chemical equation A chemical equation that has the same number of atoms of each element involved in the reaction on both sides of the equation.
Barometer A device used to measure atmospheric pressure.
Base ionization constant (Kb) The equilibrium constant for the reaction of a weak base with water.
Basic solution An aqueous solution in which the concentration of OH ion is higher than that of H3O ion; an aqueous solutionwhose pH is greater than 7.0.
Becquerel, Antoine Henri
Binary compound A compound in which only two elements are present.
Binary ionic compound An ionic compound in which one element present is a metal and the other element present is a non-metal.
Binary molecular compound A molecular compound in which only two nonmetallic elements are present.
Boiling A form of evaporation where conversion from the liquid state to the vapor state occurs within the body of the liquid through bubble formation.
Boiling point The temperature at which the vapor pressure of a liquid becomes equal to the external (atmospheric) pressure exerted on the liquid.
Bond polarity A measure of the degree of inequality in the sharing of electrons between two atoms in a chemical bond.
Bonding electrons Pairs of valence electrons that are shared between atoms in a covalent bond.
Boyle, Robert
Boyle’s law The volume of a fixed amount of a gas is inversely proportional to the pressure applied to the gas if the temperature is kept constant.
Brønsted, Johannes Nicolaus
Brønsted–Lowry acid A substance that can donate a proton (H ion) to some other substance.
Brønsted–Lowry base A substance that can accept a proton (H ion) from some other substance.
Buffer An aqueous solution containing substances that prevent major changes in solution pH when small amounts of acid or base are added it.
Calorie The amount of heat energy needed to raise the temperature of 1 gram of water by 1 degree Celsius.
Catalyst A substance that increases a chemical reaction rate without being consumed in the reaction.
Change of state A process in which a substance is changed from one physical state to another physical state.
Charles, Jacques
Charles’s law The volume of a fixed amount of gas is directly proportional to its Kelvin temperature if the pressure is kept constant.
Chemical bond The attractive force that holds two atoms together in a more complex unit.
Chemical change A process in which a substance undergoes a change in chemical composition.
Chemical equation A written statement that uses chemical symbols and chemical formulas instead of words to describe the changes that occur in a chemical reaction.

Chemical equilibrium A state in which forward and reverse chemical reactions occur simultaneously at the same rate.
Chemical formula A notation made up of the chemical symbols of the elements present in a compound and numerical subscripts (located to the right of each chemical symbol) that indicate the number of atoms of each element present in a molecule of the compound.
Chemical property A characteristic of a substance that describes the way the substance undergoes or resists change to form a new substance.
Chemical reaction A process in which at least one new substance is produced as a result of chemical change.
Chemical reaction rate The rate at which reactants are consumed or products produced in a given time period in a chemical reaction. 
Chemical symbol A oneor two-letter designation for an element derived from the element’s name.
Chemistry The field of study concerned with the characteristics, composition, and transformations of matter.
Colligative property A physical property of a solution that depends only on the number (concentration) of solute particles (molecules or ions) present in a given quantity of solvent and not on their chemical identities.
Collision theory A set of statements that give the conditions necessary for a chemical reaction to occur.
Combination reaction A chemical reaction in which a single product is produced from two (or more) reactants.
Combined gas law The product of the pressure and volume of a fi xed amount of gas is directly proportional to its Kelvin temperature.
Combustion reaction A chemical reaction between a substance and oxygen (usually from air) that proceeds with the evolution of heat and light (usually from a flame), Compound A pure substance that can be broken down into two or more simpler pure substances by chemical means.
Compressibility A measure of the change in volume in a sample of matter resulting from a pressure change.
Concentratesolution A solution that contains a large amount of solute relative to the 
amount that could dissolve. Concentration The amount of solute present in a specifi ed amount of solution. Condensation reaction A chemical reaction in which two molecules combine to form a larger one while liberating 
a small molecule, usually water. Condensed structural formula A structural formula that uses groupings of atoms, in which central atoms and the atoms connected to them are written as a group, to convey molecular 
structural information. Conjugate acid The species formed when a proton (H ion) is added to a Brønsted–Lowry base. Conjugate acid–base pair Two substances, one an acid and one a base, that differ from each other 
through the loss or gain of a proton (H ion). Conjugate base The species formed that remains when a proton (H ion) is removed from a Brønsted–Lowry acid. Conversion factor A ratio that specifies how one unit of 
measurement is related to another unit of measurement. Coordinate covalent bond A covalent bond in which both electrons of a shared pair come from one of the two atoms involved in the bond. Covalent bond A 
chemical bond formed through the sharing of one or more pairs of electrons between two atoms; a chemical bond resulting from two nuclei attracting the same shared electrons. Dalton, John Dalton’s law of partial 
pressures The total pressure exerted by a mixture of gases is the sum of the partial pressures of the individual gases present.–174 Decomposition reaction A chemical reaction in which a single reactant is converted 
into two (or more) simpler substances (elements or compounds). Dehydration reaction A chemical reaction in which the components of water (H and OH) are removed from a single reactant or from two reactants (H from one 
and OH from the other).–412 Delocalized bond A covalent bond in which electrons are shared among more than two atoms. Density The ratio of the mass of an object to the volume occupied by that object. Diatomic molecule 
A molecule that contains two atoms. Dilute solution A solution that contains a small amount of solute relative to the amount that could dissolve. Dilution The process in which more solvent is added to a solution in 
order to lower its concentration. Dimensional analysis A general problemsolving method in which the units associated with numbers are used as a guide in setting up calculations. Dipole–dipole interaction An 
intermolecular force that occurs between polar molecules. 10 Diprotic acid An acid that supplies two protons (H ions) per molecule during an acid–base reaction. Dissociation The process in which individual positive 
and negative ions are released from an ionic compound that is dissolved in solution. Distinguishing electron The last electron added to the electron confi guration for an element when electron subshells are filled 
in order of increasing energy. Double covalent bond A covalent bond in which two atoms shared two pairs of electrons. Double-replacement reaction A chemical reaction in which two substances exchange parts with one 
another and form two different substances. Electrolyte A substance whose aqueous solution conducts electricity. Electron A subatomic particle that possesses a negative electrical charge. Electron configuration A 
statement of how many electrons an atom has in each of its electron subshells. Electron orbital A region of space within an electron subshell where an electron with a specifi c energy is most likely to be found. 
Electron shell A region of space about a nucleus that contains electrons that have approximately the same energy and that spend most of their time approximately the same distance from the nucleus. Electron subshell 
A region of space within an electron shell that contains electron that have the same energy. Electronegativity A measure of the relative attraction that an atom has for the shared electrons in a bond.–123 
Electrostatic interaction An attraction or repulsion that occurs between charged particles. Element A pure substance that cannot be broken down into simpler pure substances by ordinary chemical means such as a 
chemical reaction, an electric current, heat, or a beam of light; a pure substance in which all atoms present have the same atomic number. Endothermic change of state A change of state in which heat energy is 
absorbed. Endothermic chemical reaction A chemical reaction in which a continuous input of energy is needed for the reaction to occur. Equation coefficient A number that is placed to the left of a chemical formula 
of a substance in a chemical equation that changes the amount, but not the identity, of the substance. Equilibrium A condition in which two opposite processes take place at the same rate. Equilibrium constant A nu
merical value that characterizes the relationship between the concentrations of reactants and products in a system at chemical equilibrium. Equilibrium position A qualitative indication of the relative amounts 
of reactants and products present when a chemical reaction reaches equilibrium. Equivalent The molar amount of an ion needed to supply one mole of positive or negative charge. Evaporation The process in which 
molecules escape from the liquid phase to the gas phase. Exact number A number whose value has no uncertainty associated with it. Exothermic change of state A change of state in which heat energy is given off. 
Exothermic chemical reaction A chemical reaction in which energy is released as the reaction occurs. Expanded structural formula A structural formula that shows all atoms in a molecule and all bonds connecting the 
atoms. Formula mass The sum of the atomic masses of all the atoms represented in the chemical formula of a substance. Formula unit The smallest whole-number repeating ratio of ions present in an ionic compound that 
results in charge neutrality. Gas The physical state characterized by an indefi nite shape and an indefi nite volume; the physical state characterized by a complete dominance of kinetic energy (disruptive forces) 
over potential energy (cohesive forces). Gas law A generalization that describes in mathematical terms the relationships among the amount, pressure, temperature, and volume of a gas. Gram The base unit of mass in the 
metric system. Group A vertical column of elements in the periodic table. Halogen A general name for any element in Group VIIA of the periodic table. Henderson–Hasselbalch equation, buffer systems and.–278 Henry, 
William Henry’s law The amount of gas that will dissolve in a liquid at a given temperature is directly proportional to the partial pressure of the gas above the liquid. Heteroatomic molecule A molecule in which two 
or more kinds of atoms are present. Heterogeneous mixture A mixture that contains visibly different phases (parts), each of which has different properties. Homoatomic molecule A molecule in which all atoms present are 
of the same kind. Homogeneous mixture A mixture that contains only one visibly distinct phase (part), which has uniform properties throughout. 11 Hydrogen bond An extra strong dipole-dipole interaction between a 
hydrogen atom covalently bonded to a small, very electronegative element (F, O, or N) and a lone pair of electrons on another small, very electronegative element (F, O, or N). Hydrolysis reaction The reaction of 
a salt with water to produce hydronium ion or hydroxide ion or both; the reaction of a compound with H2O, in which the compound splits into two or more fragments as the elements of water (H— and —OH) are added to 
the compound. Ideal gas law A gas law that describes the relationship among the four variables temperature, pressure, volume, and molar amount for a gaseous substance at a given set of conditions. Inexact number A 
number whose value has a degree of uncertainty associated with it. Inner transition element An element located in the f area of the periodic table. Inorganic chemistry The study of all substances other than 
hydrocarbons and their Intermolecular force An attractive force that acts between a molecule and another molecule. Ion An atom (or group of atoms) that is electrically charged as a result of loss or gain of 
electrons. Ion pair The electron and positive ion that are produced during an interaction between an atom or a molecule and ionizing radiation. Ionic bond A chemical bond formed through the transfer of one or more 
electrons from one atom or group of atoms to another atom or group of atoms. Ionic compound A compound in which ionic bonds are present. Ionization The process in which individual positive and negative ions are 
produced from a molecular compound that is dissolved in solution. Ionizing radiation Radiation with suffi cient energy to remove an electron from an atom or a molecule. Isoelectronic species An atom and an ion, or 
two ions, that have the same number and confi guration of electrons. Isotopes Atoms of an element that have the same number of protons and same number of electrons but different numbers of neutrons. Kinetic energy 
Energy that matter possesses because of particle motion. Kinetic molecular theory of matter A set of five statements used to explain the physical behavior of the three states of matter (solids, liquids, and 
gases). Le Chˆatelier’s principle If a stress (change of conditions) is applied to a system at equilibrium, the system will readjust (change equilibrium position) in the direction that best reduces the stress 
imposed on the system. Lewis, Gilbert N. Lewis structure A combination of Lewis symbols that represents either the transfer or the sharing of electrons in chemical bonds. Lewis symbol The chemical symbol of an element 
surrounded by dots equal in number to the number of valence electrons present in atoms of the element. Liquid The physical state characterized by an indefi nite shape and a defi nite volume; the physical state 
characterized by potential energy (cohesive forces) and kinetic energy (disruptive forces) of about the same magnitude. Liter The base unit of volume in the metric system. London, Fritz. London force A weak 
temporary intermolecular force that occurs between an atom or molecule (polar or nonpolar) and another atom or molecule (polar or nonpolar). Lowry, Thomas Martin Mass A measure of the total quantity of matter in an 
object. Mass number The sum of the number of protons and the number of neutrons in the nucleus of an atom. Mass–volume percent The mass of solute in a solution (in grams) divided by the total volume of solution (in 
milliliters), multiplied by 100. Matter Anything that has mass and occupies space. Measurement The determination of the dimensions, capacity, quantity, or extent of something. Mendeleev, Dmitri Ivanovich Metal An 
element that has the characteristic properties of luster, thermal conductivity, electrical conductivity, and malleability. Meter The base unit of length in the metric system. Meyer, Julius Lothar Mixture A physical 
combination of two or more pure substances in which each substance retains its own chemical identity. Molar mass The mass, in grams, of a substance that is numerically equal to the substance’s formula mass. Molarity 
The moles of solute in a solution divided by the liters of solution. Mole 6.02 3 10 23 objects; the amount of a substance that contains as many elementary particles (atoms, molecules, or formula units) as there are 
atoms in exactly 12 grams of 12C. Molecular compound A compound in which covalent bonds are present. Molecular polarity A measure of the degree of inequality in the attraction of bonding electrons to various 
locations within a molecule. Molecule A group of two or more atoms that functions as a unit because the atoms are Monoatomic ion An ion formed from a single atom through loss or gain of electrons. 12 Monoprotic acid 
An acid that supplies one proton (H ion) per molecule during an acid-base reaction. Neutral solution An aqueous solution in which the concentrations of H3O ion and OH ion are equal; an aqueous solution whose pH is 
7.0.–265. Neutron A subatomic particle that has no charge associated with it. Noble gas A general name for any element in Group VIIIA of the periodic table; an element located in the far right column of the periodic 
table. Nonaqueous solution A solution in which a substance other than water is the solvent. Nonbonding electrons Pairs of valence electrons on an atom that are not involved in electron sharing. Nonelectrolyte A 
substance whose aqueous solution does not conduct electricity. Nonmetal An element characterized by the absence of the properties of luster, thermal conductivity, electrical conductivity, and malleability. 
Nonoxidation–reduction chemical reaction A chemical reaction in which there is no transfer of electrons from one reactant to another reactant. Nonpolar covalent bond A covalent bond in which there is equal sharing of 
electrons between atoms. Nonpolar molecule A molecule in which there is a symmetrical distribution of electron Normal boiling point The temperature at which a liquid boils under a pressure of 760 mm Hg. Nucleus The 
small, dense, positively charged center of an atom. Octet rule In forming compounds, atoms of elements lose, gain, or share electrons in such a way as to produce a noble-gas electron confi guration for each of the 
atoms involved. Orbital diagram A diagram that shows how many electrons an atom has in each of its occupied electron orbitals. Organic chemistry The study of hydrocarbons and their derivatives. Osmolarity The 
product of a solution’s molarity and the number of particles produced per formula unit if the solute dissociates. Osmosis The passage of solvent through a semipermeable membrane separating a dilute solution (or 
pure solvent) from a more concentrated solution. Osmotic pressure The pressure that must be applied to prevent the net fl ow of solvent through a semipermeable membrane from a solution of lower concentration to a 
solution of higher concen tration, Oxidation The process whereby a reactant in a chemical reaction loses one or more electrons. Oxidation number A number that represents the charge that an atom appears to have when 
the electrons in each bond it is participating in are assigned to the more electronegative of the two atoms involved in the bond. Oxidation–reduction chemical reaction A chemical reaction in which there is a 
transfer of electrons from one reactant to another reactant. Oxidizing agent The reactant in a redox reaction that causes oxidation of another reactant by accepting electrons from it. Pauling, Linus Carl Percent 
by mass The mass of solute in a solution divided by the total mass of solution, multiplied by 100. Percent by volume The volume of solute in a solution divided by the total volume of solution, multiplied by 100. 
Period A horizontal row of elements in the periodic table. Periodic law When elements are arranged in order of increasing atomic number, elements with similar chemical properties occur at periodic (regularly 
recurring) intervals. Periodic table A tabular arrangement of the elements in order of increasing atomic number such that elements having similar chemical properties are positioned in vertical columns. pH The 
negative logarithm of an aqueous solution’s molar hydronium ion concentration. pH scale A scale of small numbers used to specify molar hydrogen ion concentrations in aqueous solution. Physical change A process 
in which a substance changes its physical appearance but not its chemical composition. Physical property A characteristic of a substance that can be observed without changing the basic identity of the substance. 
Polar covalent bond A covalent bond in which there is unequal sharing of electrons between two atoms. Polar molecule A molecule in which there is an unsymmetrical distribution of electron charge. Polyatomic ion An ion 
formed from a group of atoms (held together by covalent bonds) through loss or gain of electrons. Polyprotic acid An acid that supplies two or more protons (H ions) per molecule in an acid-base reaction. Potential 
energy Stored energy that matter possesses as a result of its position, condition, and/or chemical composition. Pressure The force applied per unit area on an object; the force on a surface divided by the area of that 
surface. Product A substance produced as a result of a chemical reaction. Property A distinguishing characteristic of a substance that is used in its identifi cation and description. 13 Proton A subatomic particle 
that possesses a positive electrical charge. Pure substance A single kind of matter that cannot be separated into other kinds of matter by any physical means. Reactant A starting substance in a chemical reaction that 
undergoes change in the chemical reaction. Reducing agent The reactant in a redox reaction that causes reduction of another reactant by providing electrons for the other reactant to accept. Reduction The process 
whereby a reactant in a chemical reaction gains one or more electrons. Representative element An element located in the s area or the first five columns of the p area of the periodic table. Reversible chemical 
reaction A chemical reaction in which the conversion of reactants to products (the forward reaction) and the conversion of products to reactants (the reverse reaction) occur simultaneously. Rounding off The process of 
deleting unwanted (nonsignifi cant) digits from calculated numbers. Salt An ionic compound containing a metal or polyatomic ion as the positive ion and a nonmetal or polyatomic ion (except hydroxide ion) as the 
negative ion. Saturated solution A solution that contains the maximum amount of solute that can be dissolved under the conditions at which the solution exists. Scientific notation A numerical system in which numbers 
are expressed in the form A 3 10n . where A is a number with a single nonzero digit to the left of the decimal place and n is a whole number. Significant figures The digits in a measurement that are known with 
certainty plus one digit that is estimated. Single covalent bond A covalent bond in which two atoms share one pair of electrons. Single-replacement reaction A chemical reaction in which an atom or molecule replaces 
an atom or group of atoms in a compound. Solid The physical state characterized by a definite shape and a definite volume; the physical state characterized by a dominance of potential energy (cohesive forces) over 
kinetic energy (disruptive forces). Solubility The maximum amount of solute that will dissolve in a given amount of solvent under a given set of conditions. Solute A component of a solution that is present in a 
lesser amount relative to that of the solvent. Solution A homogeneous mixture of two or more substances with each substance retaining its own chemical identity. Solvent The component of a solution that is present in 
the greatest amount. Specific heat The quantity of heat energy, in calories, necessary to raise the temperature of 1 gram of a substance by 1 degree Celsius. Strong acid An acid that transfers 100%, or very nearly 
100%, of its protons (H ions) to water when in an aqueous solution, Strong base, commonly encountered. Strong electrolyte A substance that completely (or almost completely) ionizes/dissociates into ions in aqueous 
solution. Structural formula A twodimensional structural representation that shows how the various atoms in a molecule are bonded to each other. Subatomic particle A very small particle that is a building block 
for atoms. Supersaturated solution An unstable solution that temporarily contains more dissolved solute than that present in a saturated solution. Suspension A heterogeneous mixture that contains dispersed particles 
that are heavy enough that they settle out under the influence of gravity. Thermal expansion A measure of the change in volume of a sample of matter resulting from a temperature change. Torricelli, Evangelista 
Triatomic molecule A molecule that contains three atoms. Triple covalent bond A covalent bond in which two atoms share three pairs of electrons. Triprotic acid An acid that supplies three protons (H ions) per 
molecule during an acid-base reaction. Valence electron An electron in the outermost electron shell of a representative element or noble-gas element. Vapor A gas that exists at a temperature and pressure at which it 
would ordinarily be thought of as a liquid or a solid. Vapor pressure The pressure exerted by a vapor above a liquid when the liquid and vapor are in equilibrium with each Volatile substance A substance that readily 
evaporates at room temperature because of a high vapor pressure. VSEPR electron group A collection of valence electrons present in a localized region about the central atom in a molecule. VSEPR theory A set of proce
dures for predicting the molecular geometry of a molecule using the information contained in the molecule’s Lewis structure. Weak acid An acid that transfers only a small percentage of its protons (H ions) to water 
when in an aqueous solution. Weak electrolyte A substance that incompletely ionizes/dissociates into ions in aqueous solution. Weight A measure of the force exerted on an object by gravitational
f\end{document}
