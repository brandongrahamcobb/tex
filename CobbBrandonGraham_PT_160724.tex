\documentclass[roman, 12pt, draft]{article}
\title{CHEM-1060 Placement Test}
\author{Brandon Graham Cobb, Clare Williams \\14500 Twelve Mile Rd, Warren, MI 48088}
\date{\today}

\begin{document}
  \maketitle
  \tableofcontents
  \section{Acids and bases}
    There are three types of acids and bases: Arrhenius, Bronsted-Lowry and Lewis.
    An acid is essentially the compliment of a base and vice versa.
    Acids and bases react with eachother, commonly forming water.
  \section{Periodic Table}
    Matter at its smallest relevant scale to chemistry are subatomic particles.
    These include electrons, protons and neutrons.
    These subatomic particles bear a physical property where electrons are attracted to protons and vice versa.
    Neutrons are neutral.
    \\
    The periodic table is a tabular representation of elements, atoms with a defined amount of protons.
    The periodic table was created to organize information about matter.
    The periodic table can be interpreted many ways.
  \subsection{Periodicity}
    Down a column, the electron configurations of elements are similar and the electronic behavior is also similar.
    These 'groups' are given a number and a name.
    Right across a row, electronegativity, a property denoting the affinity of electrons for an element, increases.
    There are multiple other trends which follow the paths of the periodic table and these trends can be used to infer the behavior of elements.  
  \section{Mathematics}
    For basic chemistry, there is algebra, geometry and logrithms. 
  \subsection{Dimensional Analysis}
    Essentially, between two different values, the units must represent the nature of each value.
    Also, converting in between values, requires constants which can be mathematically applied to determine unknown quantities or known quantities with different units.
  \subsection{Stoichiometry}
    In a chemical equation, there are reactants and products divided by a chemical reaction arrow.
    Chemicals do not react in equimolar amounts, meaning one reactant might deplete faster than the other reactant.
    This is denoted by putting a coefficient in front of the reactant in the chemical equation.
    Once atom counts are equivalent on both sides, this is considered a balanced chemical equation. 
  \section{Glossary}

\end{document}
