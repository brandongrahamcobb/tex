\documentclass[10pt, roman]{article}
\usepackage[utf8]{inputenc}
\usepackage{hyperref}
\usepackage{xcolor}
\usepackage{amsmath}
\usepackage{enumitem}
\usepackage{fancyhdr}
\usepackage[margin=1in]{geometry}
\usepackage{graphicx}
\usepackage{array}
\usepackage{caption}
\usepackage{booktabs}
\usepackage{mhchem}
\geometry{margin=1in}

\graphicspath{{/home/spawd/Pictures/}}

\title{CHEM 1060 Placement Test Guide}
\author{
Brandon Cobb, Clare Williams \\
Academic Success Center, Macomb Community College \\
\textbf{Center Campus}: 44575 Garfield Rd, Clinton Twp, MI 48038 \\
\textbf{South Campus}: 14500 12 Mile Rd, Warren, MI 48088
}

\begin{document}

\maketitle

\tableofcontents

\section{Who What When Where Why}
{\centering https://my.macomb.edu/search/?q=placement+tests\par}
\textbf{WHO}: Macomb Community College
\textbf{WHAT}: Placement tests are available for students at Macomb Community College who have expertise in Chemistry, Physics and Foreign Lanugages.
Macomb Community College has a placement test for CHEM-1060.
To demonstrate proficiency in CHEM-1050 and place into a higher level class, a student can opt to take the placement test.
\text

\section{Testing Procedure}
Students may take this exam anytime during the posted Academic Success Center hours of operation.
No appointment is required must arrive no later than 60 minutes before closing time.
There is no fee and students are allowed 1 attempt per lifetime to complete the CHEM-1060 placement test.
Students must have photo identification and supply their Macomb student identification number.
Blank scrap paper, a scantron, a periodic table of elements and a scientific calculator will be provided with the exam.
The exam is 44 questions and must be completed in 45 minutes.

\section{Passing Score}

A minimum of 25 correct answers are required to achieve a passing score.
Students who pass the exam will be able to register for CHEM-1060 Introduction to Organic Chemisty \& Biochemistry within 2 business days.

\section{Acid/base Chemistry}
This overview will explore the definitions, properties, and strength of acids and bases, as well as the concepts of pH and neutralization reactions.
\subsection{Introduction}
An \textbf{acid} is a substance that donates a proton (H\(^+\)) to another substance.
According to the Brønsted-Lowry definition, acids are proton donors. 
The Lewis definition, on the other hand, describes acids as electron pair acceptors.
Common characteristics of acids include a sour taste, the ability to turn blue litmus paper red, and the ability to react with metals to produce hydrogen gas.
A \textbf{base} is a substance that accepts a proton (H\(^+\)) from another substance.
According to the Brønsted-Lowry definition, bases are proton acceptors.
The Lewis definition describes bases as electron pair donors.
Common characteristics of bases include a bitter taste, a slippery feel, and the ability to turn red litmus paper blue.
\subsection{Discussion}
\subsubsection{Strength}
Strong acids completely dissociate in water, releasing all of their protons.
Examples include hydrochloric acid (HCl), sulfuric acid (H\(_2\)SO\(_4\)), and nitric acid (HNO\(_3\)).
Weak acids only partially dissociate in water.
Examples include acetic acid (CH\(_3\)COOH) and citric acid (C\(_6\)H\(_8\)O\(_7\)).
Strong bases completely dissociate in water, releasing hydroxide ions (OH\(^-\)).
Examples include sodium hydroxide (NaOH) and potassium hydroxide (KOH).
Weak bases only partially dissociate in water.
Examples include ammonia (NH\(_3\)) and methylamine (CH\(_3\)NH\(_2\)).
\subsubsection{pH}
The pH scale is a measure of the acidity or basicity of an aqueous solution.
It ranges from 0 to 14, with 7 being neutral.
A pH less than 7 indicates an acidic solution, while a pH greater than 7 indicates a basic solution.
The pH is defined as the negative logarithm of the hydrogen ion concentration:
\[
\text{pH} = -\log[\text{H}^+]
\]
\subsubsection{Neutralization}
A neutralization reaction occurs when an acid and a base react to form water and a salt.
The general form of a neutralization reaction is:
\[
\text{Acid} + \text{Base} \rightarrow \text{Salt} + \text{Water}
\]
\section{Areas of the Periodic Table}
\begin{figure}[ht!]
\centering
\includegraphics[width=\textwidth]{ThePeriodicTable.pdf}
\label{overflow}
\end{figure}
\section{Balancing Chemical Equations}
\subsection{Abstract}
It can be tricky at first, but once you get the hang of it, you should be good to go!
According to the first law of thermodynamics, energy (and therefore mass) is conserved.
To reflect this, chemical equations must be balanced to ensure that the same number of atoms of each element is present on both sides of the equation.
\begin{enumerate}
    \item \textbf{Write the Unbalanced Equation:} Start by writing the skeleton equation with the correct formulas for all reactants and products. For example:
    \[
    \ce{H2 + O2 -> H2O}
    \]
    \item \textbf{Count the Atoms of Each Element:} Count the number of atoms of each element on both sides of the equation.
    \begin{itemize}
        \item Reactants: 2 hydrogen (H) atoms, 2 oxygen (O) atoms
        \item Products: 2 hydrogen (H) atoms, 1 oxygen (O) atom
    \end{itemize}
    \item \textbf{Add Coefficients to Balance Atoms:} Adjust the coefficients (numbers in front of the formulas) to balance the number of atoms of each element on both sides. Begin with the element that appears in the fewest compounds.
    \begin{itemize}
        \item To balance oxygen, we need 2 oxygen atoms in the products. Place a coefficient of 2 in front of \(\ce{H2O}\):
        \[
        \ce{H2 + O2 -> 2H2O}
        \]
        \item Now, count the atoms again:
        \begin{itemize}
            \item Reactants: 2 hydrogen (H) atoms, 2 oxygen (O) atoms
            \item Products: 4 hydrogen (H) atoms, 2 oxygen (O) atoms
        \end{itemize}
        \item To balance hydrogen, place a coefficient of 2 in front of \(\ce{H2}\):
        \[
        \ce{2H2 + O2 -> 2H2O}
        \]
    \end{itemize}
    \item \textbf{Verify the Balance:} Ensure that the number of atoms of each element is equal on both sides of the equation.
    \begin{itemize}
        \item Reactants: 4 hydrogen (H) atoms, 2 oxygen (O) atoms
        \item Products: 4 hydrogen (H) atoms, 2 oxygen (O) atoms
    \end{itemize}
    \item \textbf{Check the Coefficients:} Ensure that all coefficients are in the lowest possible ratio.
    \begin{itemize}
        \item The coefficients 2:1:2 are already in the lowest ratio.
    \end{itemize}
\end{enumerate}
The balanced chemical equation is:
\[
\ce{2H2 + O2 -> 2H2O}
\]
\section{Calculating with Avogadro's Number}
\section{Calculating with Percentages}
Calculating with percentages may seem confusing at first, but there is a simple trick to them!
\section{Collision Theory}
Collision theory is a fundamental principle in chemical kinetics that explains how chemical reactions occur and why reaction rates differ for different reactions. According to this theory, for a reaction to take place, reacting molecules must collide with sufficient energy and proper orientation.
\subsection{Introduction}

\subsubsection{Effective Collisions}

Not all collisions between reactant molecules lead to a reaction. For a collision to be effective, two main criteria must be met:
\begin{itemize}
    \item \textbf{Sufficient Energy:} The colliding molecules must possess a minimum amount of energy known as the activation energy (\(E_a\)). This energy is necessary to break bonds in the reactants and form new bonds in the products.
    \item \textbf{Proper Orientation:} The molecules must be oriented in a manner that allows the formation of new chemical bonds. Incorrect orientation can lead to ineffective collisions, even if the energy criterion is satisfied.
\end{itemize}

\subsubsection{Activation Energy}

The activation energy (\(E_a\)) is the energy barrier that must be overcome for a reaction to occur. It is the difference in energy between the reactants and the transition state, which is a high-energy, unstable arrangement of atoms that exists momentarily during the reaction process.

\[
E_a = E_{\text{transition state}} - E_{\text{reactants}}
\]

\subsubsection{Maxwell-Boltzmann Distribution}

The Maxwell-Boltzmann distribution describes the distribution of kinetic energies among molecules in a gas at a given temperature. At higher temperatures, a greater proportion of molecules have sufficient energy to overcome the activation energy barrier, leading to an increased reaction rate.

%\begin{figure}[h!]
%    \centering
%    \includegraphics[width=0.5\textwidth]{maxwell-boltzmann.png}
%    \caption{Maxwell-Boltzmann distribution of molecular energies at different temperatures.}
%    \label{fig:mb-distribution}
%\end{figure}

\subsubsection{Collision Frequency}

The rate of a chemical reaction is also influenced by the frequency of collisions between reactant molecules. Collision frequency is affected by factors such as concentration, temperature, and the physical state of the reactants. Higher concentrations and temperatures generally lead to more frequent collisions, increasing the likelihood of effective collisions.

\subsubsection{Mathematical Expression}

The rate of a chemical reaction according to collision theory can be expressed as:

\[
\text{Rate} = Z \cdot f \cdot e^{-\frac{E_a}{RT}}
\]

where:
\begin{itemize}
    \item \(Z\) is the collision frequency.
    \item \(f\) is the fraction of collisions with proper orientation.
    \item \(e^{-\frac{E_a}{RT}}\) is the fraction of collisions with sufficient energy to overcome the activation energy barrier, where \(R\) is the gas constant and \(T\) is the temperature in Kelvin.
\end{itemize}

Collision theory provides a framework for understanding how and why chemical reactions occur at different rates. By considering the energy and orientation of molecular collisions, as well as factors like temperature and concentration, collision theory helps explain the kinetic behavior of chemical reactions.

\section{Concentration Calculations}
Concentration is a measure of the amount of solute present in a given quantity of solvent or solution. It is a crucial concept in chemistry as it helps to quantify the composition of mixtures and solutions. Various units and methods are used to express and calculate concentration, including molarity, molality, and percent composition.

\subsubsection{Molarity (M)}

Molarity is defined as the number of moles of solute per liter of solution. It is represented by the symbol \( M \) and is calculated using the formula:

\[
M = \frac{\text{moles of solute}}{\text{liters of solution}}
\]

For example, if 1 mole of sodium chloride (NaCl) is dissolved in 1 liter of water, the molarity of the NaCl solution is 1 M.

\subsubsection{Molality (m)}

Molality is defined as the number of moles of solute per kilogram of solvent. It is represented by the symbol \( m \) and is calculated using the formula:

\[
m = \frac{\text{moles of solute}}{\text{kilograms of solvent}}
\]

For example, if 2 moles of glucose (C$_6$H$_{12}$O$_6$) are dissolved in 1 kilogram of water, the molality of the glucose solution is 2 m.

\subsubsection{Percent Composition}

Percent composition refers to the percentage of a solute in a solution, either by mass or by volume. It is calculated as:

\[
\text{Percent by mass} = \left( \frac{\text{mass of solute}}{\text{mass of solution}} \right) \times 100
\]

\[
\text{Percent by volume} = \left( \frac{\text{volume of solute}}{\text{volume of solution}} \right) \times 100
\]

For example, if 5 grams of ethanol is dissolved in 95 grams of water, the percent by mass of ethanol in the solution is 5\%.

Concentration is a key concept in chemistry for quantifying the composition of solutions. Understanding different ways to express and calculate concentration, such as molarity, molality, and percent composition, is essential for preparing and analyzing chemical solutions.

\section{Density}
\section{Dilution}
\section{Electrolytes}
\section{Electron Configurations}
Electron configuration is the distribution of electrons of an atom or molecule in atomic or molecular orbitals. Understanding electron configurations is essential for predicting chemical, magnetic, and optical properties of substances.
\subsection{Introduction}
\subsubsection{Pauli Exclusion Principle}

The Pauli Exclusion Principle states that no two electrons in an atom can have the same set of four quantum numbers. As a result, each orbital can hold a maximum of two electrons with opposite spins.

\subsubsection{Aufbau Principle}

The Aufbau Principle states that electrons occupy orbitals of lower energy first before filling higher energy orbitals. The order in which orbitals are filled is determined by their energy levels, which can be represented by the following sequence:

\[
1s \rightarrow 2s \rightarrow 2p \rightarrow 3s \rightarrow 3p \rightarrow 4s \rightarrow 3d \rightarrow 4p \rightarrow 5s \rightarrow 4d \rightarrow 5p \rightarrow 6s \rightarrow 4f \rightarrow 5d \rightarrow 6p \rightarrow 7s \rightarrow 5f \rightarrow 6d \rightarrow 7p
\]

\subsubsection{Hund's Rule}

Hund's Rule states that electrons will fill degenerate orbitals (orbitals of the same energy) singly first, with parallel spins, before pairing up in the orbitals.

\subsubsection{Notation}

Electron configurations are written using a notation that indicates the energy level, type of orbital, and number of electrons in those orbitals. For example, the electron configuration of carbon (\ce{C}) is:

\[
1s^2 \, 2s^2 \, 2p^2
\]

This notation indicates that carbon has two electrons in the 1s orbital, two electrons in the 2s orbital, and two electrons in the 2p orbital.

Understanding electron configurations is crucial for predicting and explaining the chemical behavior of elements. The principles of the Pauli Exclusion Principle, Aufbau Principle, and Hund's Rule guide the arrangement of electrons in atomic orbitals, providing insights into the properties and reactivity of atoms.
\section{Energy Level Transitions in Atoms}
\section{Entropy of a System}
\section{Finding Molecular Weight}
\subsection*{Introduction}
\section{Identifying Diatomic Elements}
\section{Identifying Isotopes}
\section{Identifying Types of Reactions}
\section{Ionization Energy}
\section{Lewis Structures}
\section{Molarity}
\section{Naming Ionic Compounds}
\section{Nuclide Symbols}
\subsection*{Introduction}
\section{Percent Composition}
\section{pH}
\section{Phase Diagrams}
\section{Precision vs Accuracy}
\section{Significant Figures}

\begin{enumerate}
    \item All nonzero digits are significant.
    For example, in the number 123.45, all five digits are significant.
    \item Any zeros between significant digits are also significant.
    For instance, in 1002, all four digits are significant.
    \item Leading zeros, which are zeros to the left of the first nonzero digit, are not significant.
    For example, 0.0045 has only two significant figures (4 and 5).
    \item Trailing zeros in a decimal number are significant.
    For example, 23.00 has four significant figures.
    However, in a whole number without a decimal point, trailing zeros may or may not be significant depending on whether the context implies a specific measurement precision.
    For instance, 1500 may have two, three, or four significant figures depending on the measurement context.
\end{enumerate}

\section{States of Matter}
\section{Stoichiometry Calculations}
Stoichiometric calculations involve using the balanced chemical equation to determine the amounts of reactants and products. This typically involves the following steps:
\begin{enumerate}
    \item \textbf{Convert quantities of known substances into moles.}
    \item \textbf{Use the balanced equation to set up the mole ratio.}
    \item \textbf{Use the mole ratio to calculate the moles of the desired substance.}
    \item \textbf{Convert moles of the desired substance back into the required units.}
\end{enumerate}
\textbf{Example Calculation}
Consider the reaction of hydrogen and oxygen to form water:
\[
\ce{2H2 + O2 -> 2H2O}
\]
Suppose we have 4 moles of \(\ce{H2}\) and an excess of \(\ce{O2}\). How many moles of water will be produced?
\begin{enumerate}
    \item \textbf{Convert the given quantity to moles (if necessary):}
    We are already given moles of \(\ce{H2}\) (4 moles).
    \item \textbf{Use the balanced equation to set up the mole ratio:}
    According to the equation, 2 moles of \(\ce{H2}\) produce 2 moles of \(\ce{H2O}\).
    \item \textbf{Calculate the moles of \(\ce{H2O}\):}
    \[
    4 \text{ moles } \ce{H2} \times \frac{2 \text{ moles } \ce{H2O}}{2 \text{ moles } \ce{H2}} = 4 \text{ moles } \ce{H2O}
    \]
    \item \textbf{Convert to the desired units (if necessary):}
    The result is already in moles, so no further conversion is needed.
\end{enumerate}
\section{Types of Bonding}
Chemical bonding is the process by which atoms combine to form compounds.
The primary types of chemical bonding include ionic, covalent, and metallic bonds.
Each type of bond arises from different interactions between atoms and has distinct properties.

\subsubsection*{Ionic Bonding}

Ionic bonding occurs when electrons are transferred from one atom to another, resulting in the formation of ions.
This type of bond typically forms between metals and nonmetals.
Metals tend to lose electrons and become positively charged cations, while nonmetals gain electrons to become negatively charged anions.
The electrostatic attraction between oppositely charged ions holds them together in an ionic compound.
A classic example of ionic bonding is found in sodium chloride (NaCl), where sodium (Na) donates an electron to chlorine (Cl), resulting in Na\(^+\) and Cl\(^-\) ions.

\subsubsection*{Covalent Bonding}

Covalent bonding involves the sharing of electrons between atoms.
This type of bond typically forms between nonmetals.
When two atoms share one or more pairs of electrons, they are held together by the mutual attraction of the shared electrons to the nuclei of both atoms.
Covalent bonds can be single, double, or triple, depending on the number of shared electron pairs.
For example, in a water molecule (H\(_2\)O), each hydrogen atom shares one electron with the oxygen atom, resulting in two single covalent bonds.

There are two main types of covalent bonds: polar and nonpolar.
In a polar covalent bond, the electrons are shared unequally between the atoms due to differences in electronegativity, causing a partial positive charge on one atom and a partial negative charge on the other.
Water (H\(_2\)O) is an example of a molecule with polar covalent bonds.
In a nonpolar covalent bond, the electrons are shared equally, as seen in molecules like oxygen (O\(_2\)) and nitrogen (N\(_2\)).

\subsubsection*{Metallic Bonding}

Metallic bonding occurs between metal atoms.
In this type of bond, the valence electrons are not bound to any particular atom and can move freely throughout the entire structure.
This "sea of electrons" allows metals to conduct electricity and heat, and gives them their characteristic malleability and ductility.
The electrons' mobility also contributes to the strength and cohesion of metallic bonds.
An example of metallic bonding can be observed in a piece of copper (Cu) or iron (Fe).

\subsubsection*{Intermolecular Forces}

In addition to these primary types of bonds, there are intermolecular forces that play a crucial role in the behavior of molecules.
These forces, which include hydrogen bonding, dipole-dipole interactions, and London dispersion forces, are weaker than ionic, covalent, and metallic bonds but are significant in determining the properties of substances.
For instance, hydrogen bonding, which occurs when a hydrogen atom covalently bonded to a highly electronegative atom (like oxygen, nitrogen, or fluorine) is attracted to another electronegative atom, is responsible for the high boiling point of water compared to other molecules of similar size.

\subsubsection*{Conclusion}

Understanding the types of chemical bonding is fundamental to studying and predicting the properties and behaviors of different substances.
Each type of bond contributes to the stability and formation of the vast array of compounds and materials we encounter in chemistry.

\section{Understanding Graphical Data}
\section{Unit Conversions}
\subsection{Introduction}
\begin{minipage}{0.48\textwidth}
\centering
\begin{tabular}{|c|}
\hline
\textbf{Length Conversion Factors} \\
\hline
1 Angstrom (Å) = 0.1 nanometers (nm) \\
1 inch (in) = 2.54 centimeters (cm) \\
1 foot (ft) = 0.3048 meters (m) \\
1 mile (mi) = 1.60934 kilometers (km) \\
\hline/
\end{tabular}

\vspace{20pt}

\begin{tabular}{|c|c|c|}
\hline
\multicolumn{3}{|c|}{\textbf{Common Metric Prefixes}} \\
\hline
Prefix & Symbol & Factor \\
\hline
Tera & T & $10^{12}$ \\
Giga & G & $10^{9}$ \\
Mega & M & $10^{6}$ \\
Kilo & k & $10^{3}$ \\
Hecto & h & $10^{2}$ \\
Deka & da & $10^{1}$ \\
\hline
(Base Unit) &  & $10^{0}$ \\
\hline
Deci & d & $10^{-1}$ \\
Centi & c & $10^{-2}$ \\
Milli & m & $10^{-3}$ \\
Micro & µ & $10^{-6}$ \\
Nano & n & $10^{-9}$ \\
Pico & p & $10^{-12}$ \\
F/emto & f & $10^{-15}$ \\
Atto & a & $10^{-18}$ \\
\hline
\end{tabular}

\vspace{20pt}

\begin{tabular}{|c|}
\hline
\textbf{Energy Conversion Factors} \\
\hline
1 calorie (cal) = 4.184 joules (J) \\
1 kilocalorie (kcal) = 4184 joules (J) \\
\hline
\end{tabular}
\end{minipage}
\hfill
\begin{minipage}{0.48\textwidth}
\centering
\begin{tabular}{|c|}
\hline
\textbf{Volume Conversion Factors} \\
\hline
1 teaspoon (tsp) = 4.92892 milliliters (mL) \\
1 tablespoon (tbsp) = 14.7868 milliliters (mL) \\
1 fluid ounce (fl oz) = 29.5735 milliliters (mL) \\
1 cup = 0.236588 liters (L) \\
1 pint (pt) = 0.473176 liters (L) \\
1 quart (qt) = 0.946353 liters (L) \\
1 gallon (gal) = 3.78541 liters (L) \\
\hline
\end{tabular}

\vspace{10pt}

\begin{tabular}{|c|}
\hline
\textbf{Mass Conversion Factors} \\
\hline
1 atomic mass unit (amu) = 1 dalton (Da) \\
1 Dalton (Da) $\approx 1.66 \times 10^{-21}$ milligrams (mg) \\
1 ounce (oz) = 28.3495 grams (g) \\
1 pound (lb) = 0.453592 kilograms (kg) \\
1 tonne (ton) = 2000 pounds (lb) \\
\hline
\end{tabular}

\vspace{10pt}

\begin{tabular}{|c|}
\hline
\textbf{Pressure Conversion Factors} \\
\hline
1 atmosphere (atm) = 101.325 kilopascals (kPa) \\
1 atmosphere (atm) = 760 mm of Hg (mmHg) \\
\hline
\end{tabular}

\vspace{10pt}

\begin{tabular}{|c|c|}
\hline
\multicolumn{2}{|c|}{\textbf{Constants}} \\
\hline
Avogadro’s \# & $N_A = 6.022 \times 10^{23} \text{ mol}^{-1}$ \\
Gas Constant & $R = 8.314 \frac{J}{\text{mol K}}$ \\
Speed of Light & $c = 2.998 \times 10^{8} \frac{m}{s}$ \\
Elementary Charge & $e = 1.602 \times 10^{-19} \text{ C}$ \\
Electron Mass & $m_e = 9.109 \times 10^{-31} \text{ kg}$ \\
Proton Mass & $m_p = 1.673 \times 10^{-27} \text{ kg}$ \\
\hline
\end{tabular}

\vspace{10pt}

\begin{tabular}{|c|}
\hline
\textbf{Time Conversions} \\
\hline
1 minute (min) = 60 seconds (s) \\
1 hour (hr) = 60 minutes (min) \\
1 day (day) = 24 hours (hr) \\
1 week (wk) = 7 days (day) \\
1 month (mo) $\approx 30.44$ days (day) \\
1 year (yr) $\approx 365.25$ days (day) \\
\hline
\end{tabular}
\end{minipage}

\subsection{Introduction}
\newpage
\subsection{notes}




no Graham
/both campus addresses
what is chem 1060 paragraph first sentence
prerequisite course -> chem 1050
remove calif test
Image not figure
change to academic success center
also provided with a scantron
link to macomb website
combine finding molecular weight and elements
get rid of dimensional analysis
dim anal take up whole page remove captions

readd reactants and products definition in the types of chemical reactions section


linear ph scal for 1060 
follow naming scheme
\newpage
\section{Glossary}
\begin{minipage}[c]{0.30\textwidth}
\textbf{Acid–base indicator} A compound that exhibits different colors depending on the pH of its solution. \\
\textbf{Acid–base neutralization chemical reaction} A chemical reaction between an acid and a hydroxide base in which a salt and water are the products. \\
\textbf{Acid–base titration} A neutralization reaction in which a measured volume of an acid or a base of known concentration is completely reacted with a measured volume of a base or an acid of unknown concentration. \\
\textbf{Acid ionization constant (Ka)} The equilibrium constant for the reaction of a weak acid with water. \\
\textbf{Acidic hydrogen atom} A hydrogen atom in an acid molecule that can be transferred to a base in an acid-base reaction. \\
\textbf{Acidic solution} An aqueous solution in which the concentration of H3O ion is higher than that of OH ion; an aqueous solution whose pH is less than 7.0. \\
\textbf{Actual yield} The amount of product actually obtained from a chemical reaction. \\
\textbf{Alkali metal} A general name for any element in Group IA of the periodic table excluding hydrogen. \\
\textbf{Alkaline earth metal} A general name for any element in Group IIA of the periodic table. \\
\textbf{Aqueous solution} A solution in which water is the solvent. \\
\textbf{Arrhenius, Svante August}. \\
\textbf{Arrhenius acid} A hydrogen-containing compound that, in water, produces hydrogen ions (H ions). \\
\textbf{Arrhenius base} A hydroxide-containing compound that, in water, produces hydroxide ions (OH ions). \\
\textbf{Atom} The smallest particle of an element that can exist and still have the properties of the element. \\
\end{minipage}%
\hfill
\begin{minipage}[c]{0.30\textwidth}
\textbf{Atomic mass} The calculated average mass for the isotopes of an element expressed on a scale where 12C serves as the reference point. \\
\textbf{Atomic number} The number of protons in the nucleus of an atom. \\
\textbf{Avogadro, Amedeo}. \\
\textbf{Avogadro’s number} The name given to the numerical value $6.02 \times 10^{23}$. \\
\textbf{Balanced chemical equation} A chemical equation that has the same number of atoms of each element involved in the reaction on both sides of the equation. \\
\textbf{Barometer} A device used to measure atmospheric pressure. \\
\textbf{Base ionization constant (Kb)} The equilibrium constant for the reaction of a weak base with water. \\
\textbf{Basic solution} An aqueous solution in which the concentration of OH ion is higher than that of H3O ion; an aqueous solutionwhose pH is greater than 7.0. \\
\textbf{Becquerel, Antoine Henri}. \\
\textbf{Binary compound} A compound in which only two elements are present. \\
\textbf{Binary ionic compound} An ionic compound in which one element present is a metal and the other element present is a non-metal. \\
\textbf{Binary molecular compound} A molecular compound in which only two nonmetallic elements are present. \\
\textbf{Boiling} A form of evaporation where conversion from the liquid state to the vapor state occurs within the body of the liquid through bubble formation. \\
\textbf{Boiling point} The temperature at which the vapor pressure of a liquid becomes equal to the external (atmospheric) pressure exerted on the liquid. \\
\end{minipage}%
\hfill
\begin{minipage}[c]{0.30\textwidth}
\textbf{Bond polarity} A measure of the degree of inequality in the sharing of electrons between two atoms in a chemical bond. \\
\textbf{Bonding electrons} Pairs of valence electrons that are shared between atoms in a covalent bond. \\
\textbf{Boyle, Robert}. \\
\textbf{Boyle’s law} The volume of a fixed amount of a gas is inversely proportional to the pressure applied to the gas if the temperature is kept constant. \\
\textbf{Brønsted, Johannes Nicolaus}. \\
\textbf{Brønsted–Lowry acid} A substance that can donate a proton (H ion) to some other substance. \\
\textbf{Brønsted–Lowry base} A substance that can accept a proton (H ion) from some other substance. \\
\textbf{Buffer} An aqueous solution containing substances that prevent major changes in solution pH when small amounts of acid or base are added it. \\
\textbf{Calorie} The amount of heat energy needed to raise the temperature of 1 gram of water by 1 degree Celsius. \\
\textbf{Catalyst} A substance that increases a chemical reaction rate without being consumed in the reaction. \\
\textbf{Change of state} A process in which a substance is changed from one physical state to another physical state. \\
\textbf{Charles, Jacques}. \\
\textbf{Charles’s law} The volume of a fixed amount of gas is directly proportional to its Kelvin temperature if the pressure is kept constant. \\
\textbf{Chemical bond} The attractive force that holds two atoms together in a more complex unit. \\
\textbf{Chemical change} A process in which a substance undergoes a change in chemical composition. \\
\end{minipage}%
\newpage
\noindent \begin{minipage}[c]{0.30\textwidth}
\textbf{Chemical equation} A written statement that uses chemical symbols and chemical formulas instead of words to describe the changes that occur in a chemical reaction. \\
\textbf{Chemical equilibrium} A state in which forward and reverse chemical reactions occur simultaneously at the same rate. \\
\textbf{Chemical formula} A notation made up of the chemical symbols of the elements present in a compound and numerical subscripts (located to the right of each chemical symbol) that indicate the number of atoms of each element present in a molecule of the compound. \\
\textbf{Chemical property} A characteristic of a substance that describes the way the substance undergoes or resists change to form a new substance. \\
\textbf{Chemical reaction} A process in which at least one new substance is produced as a result of chemical change. \\
\textbf{Chemical reaction rate} The rate at which reactants are consumed or products produced in a given time period in a chemical reaction. \\
\textbf{Chemical symbol} A oneor two-letter designation for an element derived from the element’s name. \\
\textbf{Chemistry} The field of study concerned with the characteristics, composition, and transformations of matter. \\
\textbf{Colligative property} A physical property of a solution that depends only on the number (concentration) of solute particles (molecules or ions) present in a given quantity of solvent and not on their chemical identities. \\
\textbf{Collision theory} A set of statements that give the conditions necessary for a chemical reaction to occur. \\
\textbf{Combination reaction} A chemical reaction in which a single product is produced from two (or more) reactants. \\
\end{minipage}%
\hfill
\begin{minipage}[c]{0.30\textwidth}
\textbf{Combined gas law} The product of the pressure and volume of a fi xed amount of gas is directly proportional to its Kelvin temperature. \\
\textbf{Combustion reaction} A chemical reaction between a substance and oxygen (usually from air) that proceeds with the evolution of heat and light (usually from a flame). \\
\textbf{Compound} A pure substance that can be broken down into two or more simpler pure substances by chemical means. \\
\textbf{Compressibility} A measure of the change in volume in a sample of matter resulting from a pressure change. \\
\textbf{Concentratesolution} A solution that contains a large amount of solute relative to the amount that could dissolve. \\
\textbf{Concentration} The amount of solute present in a specifi ed amount of solution. \\
\textbf{Condensation reaction} A chemical reaction in which two molecules combine to form a larger one while liberating a small molecule, usually water. \\
\textbf{Condensed structural formula} A structural formula that uses groupings of atoms, in which central atoms and the atoms connected to them are written as a group, to convey molecular structural information. \\
\textbf{Conjugate acid} The species formed when a proton (H ion) is added to a Brønsted–Lowry base. \\
\textbf{Conjugate acid–base pair} Two substances, one an acid and one a base, that differ from each other through the loss or gain of a proton (H ion). \\
\textbf{Conjugate base} The species formed that remains when a proton (H ion) is removed from a Brønsted–Lowry acid. \\
\end{minipage}%
\hfill
\begin{minipage}[c]{0.30\textwidth}
\textbf{Conversion factor} A ratio that specifies how one unit of measurement is related to another unit of measurement. \\
\textbf{Coordinate covalent bond} A covalent bond in which both electrons of a shared pair come from one of the two atoms involved in the bond. \\
\textbf{Covalent bond} A chemical bond formed through the sharing of one or more pairs of electrons between two atoms; a chemical bond resulting from two nuclei attracting the same shared electrons. \\
\textbf{Dalton, John} Dalton’s law of partial pressures The total pressure exerted by a mixture of gases is the sum of the partial pressures of the individual gases present. \\
\textbf{Decomposition reaction} A chemical reaction in which a single reactant is converted into two (or more) simpler substances (elements or compounds). \\
\textbf{Dehydration reaction} A chemical reaction in which the components of water (H and OH) are removed from a single reactant or from two reactants (H from one and OH from the other). \\
\textbf{Delocalized bond} A covalent bond in which electrons are shared among more than two atoms. \\
\textbf{Density} The ratio of the mass of an object to the volume occupied by that object. \\
\textbf{Diatomic molecule} A molecule that contains two atoms. \\
\textbf{Dilute solution} A solution that contains a small amount of solute relative to the amount that could dissolve. \\
\textbf{Dilution} The process in which more solvent is added to a solution in order to lower its concentration. \\
\end{minipage}%
\newpage
\noindent \begin{minipage}[c]{0.30\textwidth}
\textbf{Dimensional analysis} A general problemsolving method in which the units associated with numbers are used as a guide in setting up calculations. \\
\textbf{Dipole–dipole interaction} An intermolecular force that occurs between polar molecules. \\
\textbf{Diprotic acid} An acid that supplies two protons (H ions) per molecule during an acid–base reaction. \\
\textbf{Dissociation} The process in which individual positive and negative ions are released from an ionic compound that is dissolved in solution. \\
\textbf{Distinguishing electron} The last electron added to the electron confi guration for an element when electron subshells are filled in order of increasing energy. \\
\textbf{Double covalent bond} A covalent bond in which two atoms shared two pairs of electrons. \\
\textbf{Double-replacement reaction} A chemical reaction in which two substances exchange parts with one another and form two different substances. \\
\textbf{Electrolyte} A substance whose aqueous solution conducts electricity. \\
\textbf{Electron} A subatomic particle that possesses a negative electrical charge. \\
\textbf{Electron configuration} A statement of how many electrons an atom has in each of its electron subshells. \\
\textbf{Electron orbital} A region of space within an electron subshell where an electron with a specifi c energy is most likely to be found. \\
\textbf{Electron shell} A region of space about a nucleus that contains electrons that have approximately the same energy and that spend most of their time approximately the same distance from the nucleus. \\
\textbf{Electron subshell} A region of space within an electron shell that contains electron that have the same energy. \\
\end{minipage}%
\hfill
\begin{minipage}[c]{0.30\textwidth}
\textbf{Electronegativity} A measure of the relative attraction that an atom has for the shared electrons in a bond. \\
\textbf{Electrostatic interaction} An attraction or repulsion that occurs between charged particles. \\
\textbf{Element} A pure substance that cannot be broken down into simpler pure substances by ordinary chemical means such as a chemical reaction, an electric current, heat, or a beam of light; a pure substance in which all atoms present have the same atomic number. \\
\textbf{Endothermic change of state} A change of state in which heat energy is absorbed. \\
\textbf{Endothermic chemical reaction} A chemical reaction in which a continuous input of energy is needed for the reaction to occur. \\
\textbf{Equation coefficient} A number that is placed to the left of a chemical formula of a substance in a chemical equation that changes the amount, but not the identity, of the substance. \\
\textbf{Equilibrium} A condition in which two opposite processes take place at the same rate. \\
\textbf{Equilibrium constant} A numerical value that characterizes the relationship between the concentrations of reactants and products in a system at chemical equilibrium. \\
\textbf{Equilibrium position} A qualitative indication of the relative amounts of reactants and products present when a chemical reaction reaches equilibrium. \\
\textbf{Equivalent} The molar amount of an ion needed to supply one mole of positive or negative charge. \\
\textbf{Evaporation} The process in which molecules escape from the liquid phase to the gas phase. \\
\textbf{Exact nuber} A number whose value has no uncertainty associated with it. \\
\end{minipage}%
\hfill
\begin{minipage}[c]{0.30\textwidth}
\textbf{Exothermic change of state} A change of state in which heat energy is given off. \\
\textbf{Exothermic chemical reaction} A chemical reaction in which energy is released as the reaction occurs. \\
\textbf{Expanded structural formula} A structural formula that shows all atoms in a molecule and all bonds connecting the atoms. \\
\textbf{Formula mass} The sum of the atomic masses of all the atoms represented in the chemical formula of a substance. \\
\textbf{Formula unit} The smallest whole-number repeating ratio of ions present in an ionic compound that results in charge neutrality. \\
\textbf{Gas} The physical state characterized by an indefi nite shape and an indefi nite volume; the physical state characterized by a complete dominance of kinetic energy (disruptive forces) over potential energy (cohesive forces). \\
\textbf{Gas law} A generalization that describes in mathematical terms the relationships among the amount, pressure, temperature, and volume of a gas. \\
\textbf{Gram} The base unit of mass in the metric system. \\
\textbf{Group} A vertical column of elements in the periodic table. \\
\textbf{Halogen} A general name for any element in Group VIIA of the periodic table. \\
\textbf{Henderson–Hasselbalch equation}
\textbf{Henry, William}
\textbf{Henry’s law} The amount of gas that will dissolve in a liquid at a given temperature is directly proportional to the partial pressure of the gas above the liquid. \\
\textbf{Heteroatomic molecule} A molecule in which two or more kinds of atoms are present. \\
\textbf{Heterogeneous mixture} A mixture that contains visibly different phases (parts), each of which has different properties. \\
\end{minipage}%
\newpage
\noindent \begin{minipage}[c]{0.30\textwidth}
\textbf{Homoatomic molecule} A molecule in which all atoms present are of the same kind. \\
\textbf{Homogeneous mixture} A mixture that contains only one visibly distinct phase (part), which has uniform properties throughout. \\
\textbf{Hydrogen bond} An extra strong dipole-dipole interaction between a hydrogen atom covalently bonded to a small, very electronegative element (F, O, or N) and a lone pair of electrons on another small, very electronegative element (F, O, or N). \\
\textbf{Hydrolysis reaction} The reaction of a salt with water to produce hydronium ion or hydroxide ion or both; the reaction of a compound with H2O, in which the compound splits into two or more fragments as the elements of water (H— and —OH) are added to the compound. \\
\textbf{Ideal gas law} A gas law that describes the relationship among the four variables temperature, pressure, volume, and molar amount for a gaseous substance at a given set of conditions. \\
\textbf{Inexact number} A number whose value has a degree of uncertainty associated with it. \\
\textbf{Inner transition element} An element located in the f area of the periodic table. \\
\textbf{Inorganic chemistry} The study of all substances other than hydrocarbons and their
\textbf{Intermolecular force} An attractive force that acts between a molecule and another molecule. \\
\textbf{Ion} An atom (or group of atoms) that is electrically charged as a result of loss or gain of electrons. \\
\textbf{Ion pair} The electron and positive ion that are produced during an interaction between an atom or a molecule and ionizing radiation. \\
\end{minipage}%
\hfill
\begin{minipage}[c]{0.30\textwidth}
\textbf{Ionic bond} A chemical bond formed through the transfer of one or more electrons from one atom or group of atoms to another atom or group of atoms. \\
\textbf{Ionic compound} A compound in which ionic bonds are present. \\
\textbf{Ionization} The process in which individual positive and negative ions are produced from a molecular compound that is dissolved in solution. \\
\textbf{Ionizing radiation} Radiation with suffi cient energy to remove an electron from an atom or a molecule. \\
\textbf{Isoelectronic species} An atom and an ion, or two ions, that have the same number and confi guration of electrons. \\
\textbf{Isotopes} Atoms of an element that have the same number of protons and same number of electrons but different numbers of neutrons. \\
\textbf{Kinetic energy} Energy that matter possesses because of particle motion. \\
\textbf{Kinetic molecular theory of matter} A set of five statements used to explain the physical behavior of the three states of matter (solids, liquids, and gases). \\
\textbf{Le Chatelier’s principle} If a stress (change of conditions) is applied to a system at equilibrium, the system will readjust (change equilibrium position) in the direction that best reduces the stress imposed on the system. \\
\textbf{Lewis, Gilbert N}. \\
\textbf{Lewis structure} A combination of Lewis symbols that represents either the transfer or the sharing of electrons in chemical bonds. \\
\textbf{Lewis symbol} The chemical symbol of an element surrounded by dots equal in number to the number of valence electrons present in atoms of the element. \\
\end{minipage}%
\hfill
\begin{minipage}[c]{0.30\textwidth}
\textbf{Liquid} The physical state characterized by an indefi nite shape and a defi nite volume; the physical state characterized by potential energy (cohesive forces) and kinetic energy (disruptive forces) of about the same magnitude. \\
\textbf{Liter} The base unit of volume in the metric system. \\
\textbf{London, Fritz}. \\
\textbf{London force} A weak temporary intermolecular force that occurs between an atom or molecule (polar or nonpolar) and another atom or molecule (polar or nonpolar). \\
\textbf{Lowry, Thomas Martin Mass} A measure of the total quantity of matter in an object. \\
\textbf{Mass number} The sum of the number of protons and the number of neutrons in the nucleus of an atom. \\
\textbf{Mass–volume percent} The mass of solute in a solution (in grams) divided by the total volume of solution (in milliliters), multiplied by 100. \\
\textbf{Matter} Anything that has mass and occupies space. \\
\textbf{Measurement} The determination of the dimensions, capacity, quantity, or extent of something. \\
\textbf{Mendeleev, Dmitri Ivanovich}. \\
\textbf{Metal} An element that has the characteristic properties of luster, thermal conductivity, electrical conductivity, and malleability. \\
\textbf{Meter} The base unit of length in the metric system. \\
\textbf{Meyer, Julius Lothar}. \\
\textbf{Mixture} A physical combination of two or more pure substances in which each substance retains its own chemical identity. \\
\end{minipage}%
\newpage
\noindent \begin{minipage}[c]{0.30\textwidth}
\textbf{Molar mass} The mass, in grams, of a substance that is numerically equal to the substance’s formula mass. \\
\textbf{Molarity} The moles of solute in a solution divided by the liters of solution. \\
\textbf{Mole} $6. \\02 \times 10^{23}$ objects; the amount of a substance that contains as many elementary particles (atoms, molecules, or formula units) as there are atoms in exactly 12 grams of 12C. \\
\textbf{Molecular compound} A compound in which covalent bonds are present. \\
\textbf{Molecular polarity} A measure of the degree of inequality in the attraction of bonding electrons to various locations within a molecule. \\
\textbf{Molecule} A group of two or more atoms that functions as a unit because the atoms are 
\textbf{Monoatomic ion} An ion formed from a single atom through loss or gain of electrons
\textbf{Monoprotic acid} An acid that supplies one proton (H ion) per molecule during an acid-base reaction. \\
\textbf{Neutral solution} An aqueous solution in which the concentrations of H3O ion and OH ion are equal; an aqueous solution whose pH is 7. \\0. \\
\textbf{Neutron} A subatomic particle that has no charge associated with it. \\
\textbf{Noble gas} A general name for any element in Group VIIIA of the periodic table; an element located in the far right column of the periodic table. \\
\textbf{Nonaqueous solution} A solution in which a substance other than water is the solvent. \\
\textbf{Nonbonding electrons} Pairs of valence electrons on an atom that are not involved in electron sharing. \\
\textbf{Nonelectrolyte} A substance whose aqueous solution does not conduct electricity. \\
\end{minipage}%
\hfill
\begin{minipage}[c]{0.30\textwidth}
\textbf{Nonmetal} An element characterized by the absence of the properties of luster, thermal conductivity, electrical conductivity, and malleability. \\
\textbf{Nonoxidation–reduction chemical reaction} A chemical reaction in which there is no transfer of electrons from one reactant to another reactant. \\
\textbf{Nonpolar covalent bond} A covalent bond in which there is equal sharing of electrons between atoms. \\
\textbf{Nonpolar molecule} A molecule in which there is a symmetrical distribution of electron Normal boiling point The temperature at which a liquid boils under a pressure of 760 mm Hg. \\
\textbf{Nucleus} The small, dense, positively charged center of an atom. \\
\textbf{Octet rule} In forming compounds, atoms of elements lose, gain, or share electrons in such a way as to produce a noble-gas electron confi guration for each of the atoms involved. \\
\textbf{Orbital diagram} A diagram that shows how many electrons an atom has in each of its occupied electron orbitals. \\
\textbf{Organic chemistry} The study of hydrocarbons and their derivatives. \\
\textbf{Osmolarity} The product of a solution’s molarity and the number of particles produced per formula unit if the solute dissociates. \\
\textbf{Osmosis} The passage of solvent through a semipermeable membrane separating a dilute solution (or pure solvent) from a more concentrated solution. \\
\textbf{Osmotic pressure} The pressure that must be applied to prevent the net fl ow of solvent through a semipermeable membrane from a solution of lower concentration to a solution of higher concentration. \\
\end{minipage}%
\hfill
\begin{minipage}[c]{0.30\textwidth}
\textbf{Oxidation} The process whereby a reactant in a chemical reaction loses one or more electrons. \\
\textbf{Oxidation number} A number that represents the charge that an atom appears to have when the electrons in each bond it is participating in are assigned to the more electronegative of the two atoms involved in the bond. \\
\textbf{Oxidation–reduction chemical reaction} A chemical reaction in which there is a transfer of electrons from one reactant to another reactant. \\
\textbf{Oxidizing agent} The reactant in a redox reaction that causes oxidation of another reactant by accepting electrons from it. \\
\textbf{Pauling, Linus Carl}. \\
\textbf{Percent by mass} The mass of solute in a solution divided by the total mass of solution, multiplied by 100. \\
\textbf{Percent by volume} The volume of solute in a solution divided by the total volume of solution, multiplied by 100. \\
\textbf{Period} A horizontal row of elements in the periodic table. \\
\textbf{Periodic law} When elements are arranged in order of increasing atomic number, elements with similar chemical properties occur at periodic (regularly recurring) intervals. \\
\textbf{Periodic table} A tabular arrangement of the elements in order of increasing atomic number such that elements having similar chemical properties are positioned in vertical columns. \\
\textbf{pH} The negative logarithm of an aqueous solution’s molar hydronium ion concentration. \\
\textbf{pH scale} A scale of small numbers used to specify molar hydrogen ion concentrations in aqueous solution. \\
\textbf{Physical change} A process in which a substance changes its physical appearance but not its chemical composition. \\
\end{minipage}%
\newpage
\noindent \begin{minipage}[c]{0.30\textwidth}
\textbf{Physical property} A characteristic of a substance that can be observed without changing the basic identity of the substance. \\
\textbf{Polar covalent bond} A covalent bond in which there is unequal sharing of electrons between two atoms. \\
\textbf{Polar molecule} A molecule in which there is an unsymmetrical distribution of electron charge. \\
\textbf{Polyatomic ion} An ion formed from a group of atoms (held together by covalent bonds) through loss or gain of electrons. \\
\textbf{Polyprotic acid} An acid that supplies two or more protons (H ions) per molecule in an acid-base reaction. \\
\textbf{Potential energy} Stored energy that matter possesses as a result of its position, condition, and/or chemical composition. \\
\textbf{Pressure} The force applied per unit area on an object; the force on a surface divided by the area of that surface. \\
\textbf{Product} A substance produced as a result of a chemical reaction. \\
\textbf{Property} A distinguishing characteristic of a substance that is used in its identifi cation and description. \\
\textbf{Proton} A subatomic particle that possesses a positive electrical charge. \\
\textbf{Pure substance} A single kind of matter that cannot be separated into other kinds of matter by any physical means. \\
\textbf{Reactant} A starting substance in a chemical reaction that undergoes change in the chemical reaction. \\
\textbf{Reducing agent} The reactant in a redox reaction that causes reduction of another reactant by providing electrons for the other reactant to accept. \\
\textbf{Reduction} The process whereby a reactant in a chemical reaction gains one or more electrons. \\
\end{minipage}%
\hfill
\begin{minipage}[c]{0.30\textwidth}
\textbf{Representative element} An element located in the s area or the first five columns of the p area of the periodic table. \\
\textbf{Reversible chemical reaction} A chemical reaction in which the conversion of reactants to products (the forward reaction) and the conversion of products to reactants (the reverse reaction) occur simultaneously. \\
\textbf{Rounding off} The process of deleting unwanted (nonsignifi cant) digits from calculated numbers. \\
\textbf{Salt} An ionic compound containing a metal or polyatomic ion as the positive ion and a nonmetal or polyatomic ion (except hydroxide ion) as the negative ion. \\
\textbf{Saturated solution} A solution that contains the maximum amount of solute that can be dissolved under the conditions at which the solution exists. \\
\textbf{Scientific notation} A numerical system in which numbers are expressed in the form A 3 10n where A is a number with a single nonzero digit to the left of the decimal place and n is a whole number. \\
\textbf{Significant figures} The digits in a measurement that are known with certainty plus one digit that is estimated. \\
\textbf{Single covalent bond} A covalent bond in which two atoms share one pair of electrons. \\
\textbf{Single-replacement reaction} A chemical reaction in which an atom or molecule replaces an atom or group of atoms in a compound. \\
\textbf{Solid} The physical state characterized by a definite shape and a definite volume; the physical state characterized by a dominance of potential energy (cohesive forces) over kinetic energy (disruptive forces). \\
\end{minipage}%
\hfill
\begin{minipage}[c]{0.30\textwidth}
\textbf{Solubility} The maximum amount of solute that will dissolve in a given amount of solvent under a given set of conditions. \\
\textbf{Solute} A component of a solution that is present in a lesser amount relative to that of the solvent. \\
\textbf{Solution} A homogeneous mixture of two or more substances with each substance retaining its own chemical identity. \\
\textbf{Solvent} The component of a solution that is present in the greatest amount. \\
\textbf{Specific heat} The quantity of heat energy, in calories, necessary to raise the temperature of 1 gram of a substance by 1 degree Celsius. \\
\textbf{Strong acid} An acid that transfers 100%, or very nearly 100%, of its protons (H ions) to water when in an aqueous solution, Strong base, commonly encountered. \\
\textbf{Strong electrolyte} A substance that completely (or almost completely) ionizes/dissociates into ions in aqueous solution. \\
\textbf{Structural formula} A twodimensional structural representation that shows how the various atoms in a molecule are bonded to each other. \\
\textbf{Subatomic particle} A very small particle that is a building block for atoms. \\
\textbf{Supersaturated solution} An unstable solution that temporarily contains more dissolved solute than that present in a saturated solution. \\
\textbf{Suspension} A heterogeneous mixture that contains dispersed particles that are heavy enough that they settle out under the influence of gravity. \\
\textbf{Thermal expansion} A measure of the change in volume of a sample of matter resulting from a temperature change. \\
\textbf{Torricelli, Evangelista}. \\
\end{minipage}%
\newpage
\noindent \begin{minipage}[c]{0.30\textwidth}
\textbf{Triatomic molecule} A molecule that contains three atoms. \\
\textbf{Triple covalent bond} A covalent bond in which two atoms share three pairs of electrons. \\
\textbf{Triprotic acid} An acid that supplies three protons (H ions) per molecule during an acid-base reaction. \\
\textbf{Valence electron} An electron in the outermost electron shell of a representative element or noble-gas element. \\
\textbf{Vapor} A gas that exists at a temperature and pressure at which it would ordinarily be thought of as a liquid or a solid. \\
\textbf{Vapor prssure} The pressure exerted by a vapor above a liquid when the liquid and vapor are in equilibrium with each. \\
\textbf{Volatile substance} A substance that readily evaporates at room temperature because of a high vapor pressure. \\
\textbf{VSEPR electron group} A collection of valence electrons present in a localized region about the central atom in a molecule. \\
\textbf{VSEPR theory} A set of procedures for predicting the molecular geometry of a molecule using the information contained in the molecule’s Lewis structure. \\
\textbf{Weak acid} An acid that transfers only a small percentage of its protons (H ions) to water when in an aqueous solution. \\
\textbf{Weak electrolyte} A substance that incompletely ionizes/dissociates into ions in aqueous solution. \\
\textbf{Weight} A measure of the force exerted on an object by gravitational. \\
\end{minipage}%
\end{document}
